% Created 2016-09-26 Mon 00:23
\documentclass[11pt]{article}
\usepackage[utf8]{inputenc}
\usepackage[T1]{fontenc}
\usepackage{fixltx2e}
\usepackage{graphicx}
\usepackage{grffile}
\usepackage{longtable}
\usepackage{wrapfig}
\usepackage{rotating}
\usepackage[normalem]{ulem}
\usepackage{amsmath}
\usepackage{textcomp}
\usepackage{amssymb}
\usepackage{capt-of}
\usepackage{hyperref}
\author{Dmitry Mukhutdinov}
\date{\today}
\title{Домашние задания по курсу машинного обучения}
\hypersetup{
 pdfauthor={Dmitry Mukhutdinov},
 pdftitle={Домашние задания по курсу машинного обучения},
 pdfkeywords={},
 pdfsubject={},
 pdfcreator={Emacs 24.5.2 (Org mode 8.3.5)},
 pdflang={English}}
\begin{document}

\maketitle
\tableofcontents


\section{Задание 1.}
\label{sec:orgheadline6}
\subsection{Постановка задачи}
\label{sec:orgheadline1}
\begin{enumerate}
\item Реализовать метрический классификатор kNN
\item Сделать кросс-валидацию; обосновать выбор числа фолдов для нее
\item Выполнить визуализацию данных
\item Настроить классификатор с 2-3 метриками и 2-3 пространственными преобразованиями
\item Для оценки качества можно использовать метрику accuracy, но лучше - f1-measure
\end{enumerate}
\subsection{Датасет}
\label{sec:orgheadline2}
Ссылка: \url{https://www.dropbox.com/s/lolwyijk22xu7na/chips.txt?dl=0}

Датасет представляет собой набор 2d-точек, разбитых на 2 класса.
\subsection{Hints}
\label{sec:orgheadline3}
TBD
\subsection{FAQ}
\label{sec:orgheadline4}
\begin{enumerate}
\item \textbf{Вопрос:}
   Как выбиралось количество фолдов для кросс-валидации? Почему именно такое?

\textbf{Ответ:}
Около 5 фолдов. Датасет очень маленький, поэтому в ином случае фолды будут
слишком мелкие. Сослаться на слайды первой лекции\footnote{Слайды первой лекции: \url{https://www.dropbox.com/sh/0fk38jg1f5ty1oz/AAD8Z_Hf8Gs6EsE3WNCBh2bWa/02-Distance.pdf?dl=0}} , стр. 18.

\item \textbf{Вопрос:}
Как правильно подобрать k - количество ближайших соседей, на которых мы
ориентируемся?

\textbf{Ответ:}
Аналогично с Leave-One-Out кросс-валидацией (Первая лекция\footnotemark[1]{}, стр. 24)

\item \textbf{Вопрос:}
Что можно делать для улучшения качества классификации, кроме подбора
гиперпараметров (т. е. числа k, выбора метрики и ядра)?

\textbf{Ответ:}
Prototype selection и распознавание аномалий (упоминалось в лекции\footnotemark[1]{}),
чтобы почистить датасет от кривых данных. Как именно это делать, не
спрашивают.

\item \textbf{Вопрос:}
   Как устроено k-d tree? Как работают kNN-запросы в нем?

\textbf{Ответ:}
Вспоминаем вычгеом и/или читаем статью\footnote{Nearest Neighbor with k-d trees: \url{http://courses.cs.washington.edu/courses/cse599c1/13wi/slides/lsh-hashkernels-annotated.pdf}}.
Обычно достаточно помахать руками и ляпнуть что-нибудь про разделение
точек по медианам при построении и отсекание квадратиков при запросе.
\end{enumerate}
\subsection{Ссылки}
\label{sec:orgheadline5}
\end{document}
